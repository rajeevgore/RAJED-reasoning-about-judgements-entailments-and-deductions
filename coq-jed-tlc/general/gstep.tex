
\documentclass[a4paper,12pt]{llncs}
\usepackage{latexsym}

\newcommand\comment[1]{}
\renewcommand\comment[1]{#1} % to print enclosed text
\renewcommand\comment[1]{} % to not print enclosed text

\begin{document}

\title{Formalised Induction Patterns for Admissibility Proofs in Coq}

\author{Jeremy E.\ Dawson\\
etc}

% \date{}
\maketitle

\abstract{
This document describes general reasoning patterns for inductive proofs
of properties such as invertibility of rules, contraction admissibility,
and describes spedifically their implementation in Coq.
}

\section{Introduction}

Proofs of properties such as admissibility 
weakening and exchange admissibility,
invertibility of certain rules,
contraction admissibility, and cut admissibility
are generally proved by induction, though there are variations
on the exact forms of induction proofs available.

For example, induction on the height (or size) of a proof,
which can sometimes require that a previously proved property such as
weakening admissibility be proved as height-preserving weakening admissibility.

Alternatively, induction on the ``proof'' itself, ie assuming that
a formula or sequent which is earlier in the proof satisfies the required
property.

It is useful to have general reasoning patterns for inductive proofs,
formalising the variety of manners of induction available.

\section{\texttt{derrec}, \texttt{derl}, \texttt{adm}}\label{s-dda}
We use the framework previously implemented in Isabelle,
described in \cite[Appendix A.1]{dawson-gore-gls},
which we recapitulate briefly here, with reference to its implementation in
Coq.

An inference rule is represented as a list of premises and a conclusion,
where the premises and conclusion are of some unspecified type.
(For all our illustrations of this work, they will be sequents whose
antecedents and succedents are lists of modal formulae).

This suggests using
\texttt{rules : list X -> X -> Prop}
(as we did initially), but we found that in Coq,
we could define the height or size of a proof object only if we
generally used \texttt{Type}, instead of \texttt{Prop}.

The predicates \texttt{derrec} and \texttt{derl} have the following meanings:

\texttt{derrec rules prems concl} means that conclusion
\texttt{concl}
is derivable from premises (or assumptions)
\texttt{prems}
using 
rules
\texttt{rules}

\texttt{derl rules ps c} means that there is a derived rule
(proof tree)
with premises (leaves)
\texttt{ps}
and conclusion (root)
\texttt{c}
constructed using rules
\texttt{rules}

Their inductive definitions also require defining
\texttt{dersrec} and \texttt{dersl}
which involve a list of conclusions.

The inductive definitions follow:
\begin{verbatim}
Inductive
derrec (X : Type) (rules : list X -> X -> Type) (prems : X -> Type)
  : X -> Type :=
    dpI : forall concl : X, prems concl -> derrec rules prems concl
  | derI : forall (ps : list X) (concl : X),
           rules ps concl ->
           dersrec rules prems ps -> derrec rules prems concl
  with dersrec (X : Type) (rules : list X -> X -> Type) 
       (prems : X -> Type) : list X -> Type :=
    dlNil : dersrec rules prems []
  | dlCons : forall (seq : X) (seqs : list X),
             derrec rules prems seq ->
             dersrec rules prems seqs -> dersrec rules prems (seq :: seqs)

Inductive
derl (X : Type) (rules : list X -> X -> Type) : list X -> X -> Type :=
    asmI : forall p : X, derl rules [p] p
  | dtderI : forall (pss ps : list X) (concl : X),
             rules ps concl -> dersl rules pss ps -> derl rules pss concl
  with dersl (X : Type) (rules : list X -> X -> Type)
         : list X -> list X -> Type :=
    dtNil : dersl rules [] []
  | dtCons : forall (ps : list X) (c : X) (pss cs : list X),
             derl rules ps c ->
             dersl rules pss cs -> dersl rules (ps ++ pss) (c :: cs)
\end{verbatim}

We get useful results like the following:
\begin{verbatim}
derl_derrec_trans
     : forall (X : Type) (rules : list X -> X -> Type) 
         (prems : X -> Type) (rps : list X) (concl : X),
       derl rules rps concl ->
       dersrec rules prems rps -> derrec rules prems concl

derrec_derl_deriv
     : forall (X : Type) (rules : list X -> X -> Type) 
         (prems : X -> Type) (concl : X),
       derrec (derl rules) prems concl -> derrec rules prems concl

derl_deriv':
  forall (X : Type) (rules : list X -> X -> Type) 
    (prems : list X) (concl : X),
  derl (derl rules) prems concl -> derl rules prems concl

derrec_derrec
     : forall (X : Type) (rules : list X -> X -> Type) 
         (prems : X -> Type) (c : X),
       derrec rules (derrec rules prems) c -> derrec rules prems c
\end{verbatim}

We also define \emph{admissible} rules:
a rule is \emph{admissible} if, assuming its premises
are derivable (from no assumptions) its conclusion is
likewise derivable.

\begin{verbatim}
Inductive
adm (X : Type) (rules : list X -> X -> Type) (ps : list X) (c : X) : Type :=
    admI : (dersrec rules (emptyT (X:=X)) ps ->
            derrec rules (emptyT (X:=X)) c) -> adm rules ps c
\end{verbatim}

Note that \texttt{adm}, unlike \texttt{derl}, is not monotonic in the
rules used.
From this we get the results 
\begin{verbatim}
adm_adm:
  forall (X : Type) (rules : list X -> X -> Type) (ps : list X) (c : X),
  adm (adm rules) ps c -> adm rules ps c
derl_sub_adm:
  forall (X : Type) (rules : list X -> X -> Type) (ps : list X) (c : X),
  derl rules ps c -> adm rules ps c
derl_adm:
  forall (X : Type) (rules : list X -> X -> Type) (ps : list X) (c : X),
  derl (adm rules) ps c -> adm rules ps c
derrec_adm:
  forall (X : Type) (rls : list X -> X -> Type) (c : X),
  derrec (adm rls) (emptyT (X:=X)) c -> derrec rls (emptyT (X:=X)) c
TODO - adm (derl rules) vs adm rules
\end{verbatim}

\section{Our induction patterns}
In previous work we defined a general lemma which enables proof of a property
of a pair of proved sequents, notably cut-admissibility.
That general lemma, simplified, applies also to proving similar properties of a
single proved sequent, such as contraction admissibility, or invertibility of
a particular rule.

A particular problem with admissibility proofs is that, when, for example,
one logic is obtained from another by adding additional rules,
and the steps in proving an admissibility result
are similar for the rules common to both logics, it requires extra effort
to formalise the proofs in a way which enables similar parts of the proofs to
be reused.

Our focus here is to set out induction patterns and relevant lemmas
useful for inductive proofs of admissibility properties which permit relevant
portions of the proof to be re-used for other calculi containing some of the
same rules.

\comment{
NB - it seems I did something pretty similar in Isabelle,
I think not written up, unless it's in the stuff we did with Jesse,
in ~jeremy/isabelle/2005/seqms/ctr.thy
see the definition of \verb|inv_stepm| as follows 

\begin{verbatim}
primrec
  (* step in invertibility ; means that, if premises (ps) derivable,
    and can apply (invs_of irls) to premises,
    then can apply (invs_of irls) to conclusion (concl) *)
  "inv_step derivs irls (ps, concl) = (set ps <= derivs -->
    (ALL p : set ps. invs_of irls p <= derivs) -->
    invs_of irls concl <= derivs)"

primrec
  (* variation on inv_step which will often be true and monotonic,
    whereby the inversions of the conclusion are derivable from the
    inversions of the premises *)
  "inv_stepm drls irls (ps, concl) = (invs_of irls concl <=
    derrec drls (set ps Un UNION (set ps) (invs_of irls)))"
\end{verbatim}

(here, \verb|invs_of irls| means the result of changing a sequent
by inverting some rule in irls)
}

The simplest example of this is probably the following:

substitution admissibility: for example, to show that if
$A \land B \vdash A$ is derivable, then $C \land D \vdash C$ is derivable.
The proof of this is obvious: you just go through the proof, changing
every $A$ to $C$ and every $B$ to $D$.

Even this example raises two points about the formal calculus used:

each individual rule must permit these substitutions

in particular the ``axiom'' rule must be $A \vdash A$ (for any formula $A$),
rather than $p \vdash p$ (for any primitive proposition $p$).

Note that in describing the implementation, the word ``rule'' implies
specific premises and conclusion, for example, one of the substitution
instances of a ``rule'' as in Table~\ref{k4-rules}.

So we formalise this argument:

we define a relation $R$ between sequents:
for example, $R s_1 s_2$ could mean that $s_2$ is obtained from 
$s_1$ by changing all occurrences of $A$ to $C$, and all $B$ to $D$
(or, applying any substitution uniformly).

Then the relevant requirement on the set of rules is that 

for any rule $\displaystyle \frac{P_1 \ldots P_n} {Q}$,
and for any $Q'$ such that $R~Q~Q'$,
there exist modified premises ${P'_1 \ldots P'_n}$
and a rule $\displaystyle \frac {P'_1 \ldots P'_n} {Q'}$.

This obviously permits changing $A$ to $C$ and $B$ to $D$ 
throughout the entire proof.

So we define \texttt{can\_trf\_rules} to express this property 
of a set of rules,
and the theorem \texttt{der\_trf} expresses the use of that property.

\begin{definition}[\texttt{can-trf-rules}]\label{def-can-trf-rules}
text of defn
\end{definition}

\begin{theorem}[\texttt{der-trf}] \label{t-der-trf}
text of thm
\end{theorem}

\begin{verbatim}
can_trf_rules = 
fun (sty : Type) (R : sty -> sty -> Type) (rules : list sty -> sty -> Type)
  (ps : list sty) (c : sty) =>
forall c' : sty,
R c c' ->
{ps' : list sty &
rules ps' c' * ForallT (fun p' : sty => {p : sty & InT p ps * R p p'}) ps'}
     : forall sty : Type,
       (sty -> sty -> Type) ->
       (list sty -> sty -> Type) -> list sty -> sty -> Type

der_trf
     : forall (sty : Type) (R : sty -> sty -> Type)
         (rules : list sty -> sty -> Type),
       (forall (ps : list sty) (c : sty),
        rules ps c -> can_trf_rules R rules ps c) ->
       forall concl : sty,
       derrec rules (emptyT (X:=sty)) concl ->
       forall concl' : sty,
       R concl concl' -> derrec rules (emptyT (X:=sty)) concl'
\end{verbatim}

\subsection{Extensions and generalizations}
We use as a running example a sequent calculus for the modal logic K4, 
with a minimal set of logical connectives ($\to$ and $\bot$)
and modal operator $\Box$.
A sequent is represented as a list for formulae on either side of the
turnstile $\vdash$.
There is no explicit weakening rule (except that the $(\Box)$ rule 
incorporates weakening (in arbitrary positions) in its conclusion,
and no explicit exchange rule.

\begin{figure}\label{k4-rules}
GIVE RULES 
\caption{Rules of our sequent calculus for K4}
\end{figure}

TO DO describe the weakening theorems,
including the gen\_ext predicate,
and the exchange theorems

In various places we found we needed extensions to the theorem.

\subsubsection{Reflexive closure of $R$}\label{s-refl-clos}
For example the K4 ($\Box$) rule allows weakening (at arbitrary points)
in the antecedent of the conclusion.  
Consider the inductive proof of exchange (which we define as
the interchanging of two sublists which are not necessarily adjacent):
in the case where we are exchanging two sublists both of which consist entirely
of formula which were ``weakened in'' in the use of the ($\Box$) rule.
Then in this case there are two rule occurrences whose conclusions are related
by the exchange relation, but which have the same premise, rather than related
premises.

So we extend the definition and theorem above to the case where the 
premises are related by the reflexive closure of the relation $R$, thus:

\begin{theorem}[\texttt{der-trf-rc}]\label{t-der-trf-rc}
\end{theorem}
\begin{verbatim}
can_trf_rules_rc = 
fun (sty : Type) (R : sty -> sty -> Type) (rules : list sty -> sty -> Type)
  (ps : list sty) (c : sty) =>
forall c' : sty,
R c c' ->
{ps' : list sty &
rules ps' c' *
ForallT (fun p' : sty => {p : sty & InT p ps * clos_reflT R p p'}) ps'}
     : forall sty : Type,
       (sty -> sty -> Type) ->
       (list sty -> sty -> Type) -> list sty -> sty -> Type

der_trf_rc
     : forall (sty : Type) (R : sty -> sty -> Type)
         (rules : list sty -> sty -> Type),
       (forall (ps : list sty) (c : sty),
        rules ps c -> can_trf_rules_rc R rules ps c) ->
       forall concl : sty,
       derrec rules (emptyT (X:=sty)) concl ->
       forall concl' : sty,
       R concl concl' -> derrec rules (emptyT (X:=sty)) concl'
\end{verbatim}

Note that it would be trivial to infer 
\texttt{can-trf-rules sty (clos-reflT R) rules ps c}
from \texttt{can-trf-rules sty R rules ps c}
if we had defined \texttt{can-trf-rules} to require that ps and c were
premises and conclusion of a rule, and in this case this definition and theorem
would not be necessary.  But some of our other results would then be more
complicated.

\subsubsection{Transitive closure of $R$}\label{s-refl-trans-clos}
The ($\Box$) rule (omitting the weakening incorporated into it) is
$\displaystyle \frac{\Gamma \Box\Gamma \vdash A}{\Box\Gamma \vdash \Box A}$.
To show the admissibility of exchange, we consider an exchange in $\Gamma$,
to give $\Gamma'$.  The corresponding rule has premise
$\Gamma' \Box\Gamma' \vdash A$ --- that is, $\Gamma \Box\Gamma \vdash A$ 
subjected to \emph{two} exchanges.
We therefore needed (and were able to) extend the 
results of \S\ref{s-refl-clos} to reflexive transitive closure, as follows

\begin{theorem}[\texttt{der-trf-rtc}]\label{t-der-trf-rtc}
\end{theorem}
\begin{verbatim}
can_trf_rules_rtc = 
fun (sty : Type) (R : sty -> sty -> Type) (rules : list sty -> sty -> Type)
  (ps : list sty) (c : sty) =>
forall c' : sty,
R c c' ->
{ps' : list sty &
rules ps' c' *
ForallT (fun p' : sty => {p : sty & InT p ps * clos_refl_transT_n1 R p p'})
  ps'}
     : forall sty : Type,
       (sty -> sty -> Type) ->
       (list sty -> sty -> Type) -> list sty -> sty -> Type

der_trf_rtc
     : forall (sty : Type) (R : sty -> sty -> Type)
         (rules : list sty -> sty -> Type),
       (forall (ps : list sty) (c : sty),
        rules ps c -> can_trf_rules_rtc R rules ps c) ->
       forall concl : sty,
       derrec rules (emptyT (X:=sty)) concl ->
       forall concl' : sty,
       clos_refl_transT_n1 R concl concl' ->
       derrec rules (emptyT (X:=sty)) concl'
\end{verbatim}

\subsubsection{Related rule is a derived rule}\label{s-der}
In proving invertibility of a rule there are some special cases.
Supposing that we are proving that if $\Gamma \vdash A \to B$ then
$\Gamma, A \vdash B$ (ie, proving that the ($\vdash\to$) rule is invertible.
Supposing then that the last rule of the proof is exactly
$\displaystyle \frac {\Gamma, A \vdash B} {\Gamma \vdash A \to B}$.
That is, given a rule $p/c$ and $c'$, we want to find $p'$ such that 
$p'/c'$ is a rule and $R\ p\ p'$, but here $c'$ is exactly $p$ (and our $p'$
will be the same also).
So here the relation between $p$ and $p'$ is the reflexive closure of $R$,
and the relationship between $p'$ and $c'$ is like a reflexive closure of
inference rule application.

Now we have defined \texttt{derl}, which is like a reflexive transitive closure
of rule application, and it is in fact easy to extend the mian result to 
derl, thus (this differs from \texttt{der-trf-derl} 
only by the insertion of the single word \texttt{derl}):

\begin{theorem}[\texttt{der-trf-derl}]\label{t-der-trf-derl}
\end{theorem}
\begin{verbatim}
der_trf_derl
     : forall (sty : Type) (R : sty -> sty -> Type)
         (rules : list sty -> sty -> Type),
       (forall (ps : list sty) (c : sty),
        rules ps c -> can_trf_rules R (derl rules) ps c) ->
       forall concl : sty,
       derrec rules (emptyT (X:=sty)) concl ->
       forall concl' : sty,
       R concl concl' -> derrec rules (emptyT (X:=sty)) concl'
\end{verbatim}

But we also need a result allowing for the fact that $p = p'$, and we can do
this:

\begin{theorem}[\texttt{der-trf-rc-derl}]\label{t-der-trf-rc-derl}
\end{theorem}
\begin{verbatim}
der_trf_rc_derl
     : forall (sty : Type) (R : sty -> sty -> Type)
         (rules : list sty -> sty -> Type),
       (forall (ps : list sty) (c : sty),
        rules ps c -> can_trf_rules_rc R (derl rules) ps c) ->
       forall concl : sty,
       derrec rules (emptyT (X:=sty)) concl ->
       forall concl' : sty,
       R concl concl' -> derrec rules (emptyT (X:=sty)) concl'
\end{verbatim}

Notice that at this point we have extended Theorem~\ref{t-der-trf}
on the one hand, to \texttt{derl} and to reflexive closure of $R$
(in Theorem~\ref{t-der-trf-rc-derl}),
and on the other hand, to reflexive transitive closure 
(in Theorem~\ref{t-der-trf-rtc}).
Can we make both extensions together?  We have tried (to some extent), and
so far have been unable.

\subsubsection{Related rule is only an admissible rule}\label{s-rc-adm}
Again we consider proving the invertibility of the ($\vdash\to$) rule,
but here, in the form, for example, where the last rule of the proof is 
$\displaystyle \frac p c = \frac {\Gamma, A \vdash B} {\Gamma \vdash A \to B}$, 
as before, but our choice of inversion which we want to prove is
$c' = {A, \Gamma \vdash B}$.
So here we need to set $p' = {\Gamma, A \vdash B}$, and the 
relationship between $p'$ and $c'$ is that $p'/c'$ is an admissible rule,
not even a derivable rule.

\begin{theorem}[\texttt{der-trf-rc-adm}]\label{t-der-trf-rc-adm}
\end{theorem}
\begin{verbatim}
der_trf_rc_adm
     : forall (sty : Type) (R : sty -> sty -> Type)
         (rules : list sty -> sty -> Type),
       (forall (ps : list sty) (c : sty),
        rules ps c -> can_trf_rules_rc R (adm rules) ps c) ->
       forall concl : sty,
       derrec rules (emptyT (X:=sty)) concl ->
       forall concl' : sty,
       R concl concl' -> derrec rules (emptyT (X:=sty)) concl'
\end{verbatim}

\subsection{Height-preserving admissibility}

In our implementation in Isabelle of the framework described in 
\S\ref{s-dda}, we defined \texttt{derl} and \texttt{derrec} as inductively
defined sets.  
We also defined a datatype representing a derivation tree, as follows
\begin{verbatim}
datatype 'a dertree = Der 'a ('a dertree list) | Unf 'a 
\end{verbatim}

(where \texttt{Unf} represents an unproved leaf).
We then defined a predicate testing whether a tree is valid according to
a set of rules (which requires no unproved leaf,
and that every \texttt{Der}-node is one of the given rules).

We then had to prove that a sequent is derivable if and only if it is the
root of a valid derivation tree.
With a derivation tree ``object'', we can define its height, size (number of
rules), etc, and prove theorems about these.

Alternatively, we could have dealt with such concepts as height, size, etc,
by defining new inductive sets (eg, sequents derivable in at most $n$ steps),
but this requires a new inductive definition for each such property which may
become of interest.

In Coq the inductive definition provides a "proof object", whose height (etc)
can be defined. This does require that the types of   
\texttt{derl}, \texttt{derrec}, etc, use \texttt{Type},
whereas our first implementation of these used \texttt{Prop}.

\begin{verbatim}
derrec : forall X : Type,
  (list X -> X -> Type) -> (X -> Type) -> X -> Type
derl : forall X : Type,
  (list X -> X -> Type) -> list X -> X -> Type 
\end{verbatim}

It also requires that relevant proved theorems 
(such as those proving some instance of \texttt{derrec \ldots \ldots \ldots})
finish with Defined rather than Qed.

We proved a theorem corresponding to Theorem~\ref{t-der-trf},
but with the additional property that the derivation tree for the 
related conclusion is no higher than that of the original conclusion.
\begin{verbatim}
der_trf_ht
     : forall (sty : Type) (R : sty -> sty -> Type)
         (rules : list sty -> sty -> Type),
       (forall (ps : list sty) (c : sty),
        rules ps c -> can_trf_rules R rules ps c) ->
       forall (concl : sty) (D : derrec rules (emptyT (X:=sty)) concl)
         (concl' : sty),
       R concl concl' ->
       {D' : derrec rules (emptyT (X:=sty)) concl' &
       derrec_height D' <= derrec_height D}
\end{verbatim}

There was also another problem: for the derivation tree in Isabelle, if you
take off the last rule, you get a list of subtrees each of which derives one
premise of the last rule.
In Coq, these subtrees all prove different things, so they have different
types, so they can't be made into a list.

So what we did instead was define \texttt{in\_dersrec}, 
which expresses that a derivation 
(of type \texttt{derrec \ldots \ldots \ldots})
is one of the derivations in an object of type 
\texttt{dersrec \ldots \ldots \ldots}).
Likewise we defined \texttt{allPder}
which represents that all the derivations represented 
in in an object of type \texttt{dersrec \ldots \ldots \ldots}).
satisfy some property.
\begin{verbatim}
in_dersrec
     : forall (X : Type) (rules : rlsT X) (prems : X -> Type) (concl : X),
       derrec rules prems concl ->
       forall concls : list X, dersrec rules prems concls -> Type

allPder
     : forall (X : Type) (rules : rlsT X) (prems : X -> Type),
       (forall x : X, derrec rules prems x -> Type) ->
       forall concls : list X, dersrec rules prems concls -> Type
\end{verbatim}

However manipulating these functions proved difficult, 
and we suspect that at some point it may become necessary to
implement a derivation tree object as we have described doing in Isabelle.
In particular there were problems which I don't understand,
 which were solved by using dependent destruction 
(which relies on some axiom, which may or may not be valid).

\section{Coq proofs about K4}
We now describe the proofs about the calculus for K4 presented in 
Figure~\ref{k4-rules}.

\subsection{Formalisation of the rules}

DISCUSS
eg \texttt{seqrule}, applies contexts (left and right, antecedent and
succedent) to premises and conclusion
\texttt{cgerule}, applies weakening (inserting formulae anywhere in the
antecedent and succedent lists) to conclusion 
\texttt{princrule}, 
\texttt{K4prrule}, ($\Box$) rule without the weakening 
\texttt{sing\_empty}, 

We then probed a weakening admissibility result, not using the framework
described here.

We next describe the proofs of further admissibility rules,
showing how we used the functions described earlier, and how the intermediate
results would be equally useful in similar proofs about a different logic,
such as S4.

\subsection{Exchange}

This lemma is meant for the classical rules, but formulated to apply to
any set of rules constructed from base rules in which antecedent satisfies
\texttt{sing\_empty} by extending premises and conclusion uniformly with
contexts.

\begin{verbatim}
exchL_std_rule: forall W (rules : rlsT (Seql W)),
  (forall ps U S, rules ps (U, S) -> sing_empty U) ->
  forall ps c, seqrule rules ps c ->
    can_trf_rules (fst_rel (swapped (T:=W))) (seqrule rules) ps c
\end{verbatim}

The relation between lists of which sublists have been swapped is
\texttt{swapped}, and so \texttt{fst\_rel (@swapped W)} is the relation between 
sequents whose antecedent lists are swapped.
Here, \texttt{sing\_empty U} means that the list $U$ is either singleton or
empty.

As an easy corollary of this, we get exchange for classical propositional
logic.

\begin{verbatim}
exchL_cpl: forall V concl,
  derrec (seqrule (@princrule V)) (@emptyT _) concl ->
     forall concl', fst_rel (@swapped _) concl concl' ->
     derrec (seqrule (@princrule V)) (@emptyT _) concl'.
\end{verbatim}

Then, adding the ($\Box$) rule we get

\begin{verbatim}
exchL_K4: forall V ps c, cgerule (@K4prrule V) ps c ->
    can_trf_rules_rtc (fst_rel (@swapped _)) (cgerule (@K4prrule V)) ps c.
\end{verbatim}

Then we put these results together, using these further lemmas: 
\begin{lemma}[\texttt{can\_trf\_rules\_rtc}]\label{l-can-trf-rules-rtc}
\texttt{can\_trf\_rules\_rtc} is monotonic in the rule set
\end{lemma}
\begin{lemma}[\texttt{can\_trf\_rules\_imp\_rtc}]\label{l-can-trf-rules-imp-rtc}
\texttt{can\_trf\_rules} R rules ps c ->
  \texttt{can\_trf\_rules\_rtc} R rules ps c
\end{lemma}

Then we use Theorem~\ref{t-der-trf-rtc} to get the result

\begin{theorem}[\texttt{exchL\_rtc, exchL}]\label{t-exchL}
If concl is derivable, then concl', got from concl by exchange[s] in the
antecedent, is derivable.
\end{theorem}

\subsection{Invertibility of the classical propositional rules}

From here on it is pretty incomplete.

First we proved the invertibility of the ($\vdash\to$) rule.

This uses Lemma~\ref{t-der-trf-rc-adm} and the following Lemma.

\begin{lemma}[\texttt{can\_trf\_ImpRinv}]\label{t-can-trf-ImpRinv}
\end{lemma}

\begin{verbatim}
Lemma can_trf_ImpRinv V ps c: @K4rules V ps c ->
  can_trf_rules_rc (seqrel (@ImpRinv V)) (adm (@K4rules V)) ps c.
\end{verbatim}

The proof of this lemma involves 5 cases, one for each rule,
each one being a separate lemma.
\footnote{as things stand right now, I need to massage these somewhat
to make them equally easily usable for other logics like S4}

For the ($\Box$) rule, the formula to be inverted can only be a formula added
in the weakening aspect of the rule, so no real difficulty there.

For the axiom rule, the only problem case is where the formula to be inverted
is the formula appearing on both sides (eg $C \to D \vdash C \to D$,
so the inversion gives $C, C \to D \vdash D$).
Here we need the related rule to be a derivable (or admissible) rule, 
rather than simply a rule of the calculus.

The ($\bot\vdash$) is easy because inversion of $C \to D$ won't change the fact 
that there is a $\bot$ in the antecedent.

The remaining two rules produce a lot of cases.
Firstly, where the last rule is also ($\vdash\to$), involving the same formula,
then the issue described in \S\ref{s-rc-adm} arises. 
Then there are a multitude of cases, according to where the principal
formulae of the last rule and the formula to be inverted appear in the contexts
of both the last rule and the inversion relation. 
So they need some effort in programming Coq Ltac tactics to handle all the
cases.

Finally, using Lemma~\ref{t-der-trf-rc-adm} we get
\begin{theorem}[\texttt{K4\_ImpRinv}]\label{t-K4-ImpRinv}
\end{theorem}
\begin{verbatim}
Theorem K4_ImpRinv V concl:
  derrec (@K4rules V) (@emptyT _) concl ->
  forall concl', seqrel (@ImpRinv V) concl concl' ->
  derrec (@K4rules V) (@emptyT _) concl'.
\end{verbatim}

Trying to generalise the above: first, I adapted the proof for
the last rule being ($\vdash\to$) and the rule being inverted
($\to\vdash$), to a case where the last rule is (in its conclusion) any
singleton on the right, and the rule being inverted is (in its conclusion) any
singleton on the left.

So I did this successfully for any such rules that are singletons (bearing in
mind that a ``rule'' like ($\vdash\to$) is actually lots of rules).
But then I have difficulty manipulating this to get the previous results.
Hopefully this won't be too hard.
Have to think about it. (the problem is that I can't simply say
rule = union of rule instances, and subsitute as the argument of
can\_trf\_rules\_rc because the ``='' really means that if there is a proof object
of one then there is a proof object of the other (I think!!)).

\begin{thebibliography}{10}

\bibitem{dawson-gore-gls}
J~E Dawson and R Gor{\'e}.
\newblock Generic Methods for Formalising Sequent Calculi
       Applied to Provability Logic.
\newblock In Proc.\ Logic for Programming, Artificial Intelligence and
Reasoning (LPAR 2010), LNCS 6397, 263-277.

\end{thebibliography}

\end{document}

